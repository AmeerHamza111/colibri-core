Colibri Core is a set of tools as well as a C++ and Python library for working with basic linguistic constructions such as n-\/grams and skipgrams (i.\+e patterns with one or more gaps, either of fixed or dynamic size) in a quick and memory-\/efficient way. At the core is the tool colibri-\/patternmodeller which allows you to build, view, manipulate and query pattern models.

In Colibri Core, text data is encoded as a compressed binary representation using a class encoding. The \hyperlink{classClassEncoder}{Class\+Encoder} and \hyperlink{classClassDecoder}{Class\+Decoder} can be used to create and decode such a class encoding. The \hyperlink{classPattern}{Pattern} class represents any n-\/gram, skip-\/gram, flexgram. These patterns can be stored in various models, such as the \hyperlink{classPatternModel}{Pattern\+Model} or it\textquotesingle{}s indexed equivalent, the \hyperlink{classIndexedPatternModel}{Indexed\+Pattern\+Model}. These are high-\/level classes built on lower-\/level containers such as \hyperlink{classPatternMap}{Pattern\+Map}. Other containers such as \hyperlink{classPatternSet}{Pattern\+Set} are available too.

Corpus data can also be read into an \hyperlink{classIndexedCorpus}{Indexed\+Corpus} class, which also acts as a reverse index for the pattern models. 